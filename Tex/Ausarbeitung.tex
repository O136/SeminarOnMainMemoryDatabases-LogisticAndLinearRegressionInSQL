\documentclass[12pt]{article}
\usepackage{amsmath}
\usepackage{graphicx}

\usepackage{hyperref}
\usepackage[latin1]{inputenc}

\title{%
  Seminar: Techniques for
implementing main memory databases \\
  \large Linear and logistic regression with SQL}

\author{Oleg Patrascu}
\date{23.12.2017}

\begin{document}
\maketitle


\paragraph{Abstract}
\begin{center}
Databases are huge containers of data and are great at aggregating it but when it comes to statistical analysis they lack a lot of features.
Most of the statistical functionality is handled by an additional layer of software like R or Matlab. 
We will see if linear or logistic regression is possible with SQL and provide their implementation and enough arguments whether an SQL implementation of these is worth, or maybe
it's not worth to perform any numerical calculations on the database side.
\end{center}

\section{Simple Linear Regression}
Simple linear regression is a statistical method that
allows us to summarize and study relationships between
two continuous (quantitative) variables: 
One variable, denoted $x$, is regarded as the predictor,
explanatory, or independent variable. 
The other variable, denoted $y$, is regarded as
the response, outcome, or dependent variable.

Given the training set $\{y_i,x_i\}^n_{i=1}$ we try to find the best coefficients of the 
equation $y_i=\beta_0+\beta_1x_i$ so that we approximate $\hat{y_i} \approx y_i$ according  
to  ordinary least squares (OLS) : conceptually the simplest and
computationally straightforward which is advantageous for SQL. 


So if we take the derivative of $\sum_{i=1} ^n (y_i - \hat{y_i})^2$ with respect to $\beta_0$ and $\beta_1$
 we get $$\beta_0 = (\bar{y} - \beta_1\bar{x})$$
 
 $$\beta_1 =\frac{\sum_{i=1} ^n (y_i - \bar{y})(x_i - \bar{x})}{\sum_{i=1} ^n (x_i - \bar{x})^2}$$

\begin{itemize}

\item With only 2 parameters to estimate : $\beta_0,\beta_1$
computationally it is not a challenge for any DBMS

\item But increasing the #parameters it will slow by noticeable
constant factor if we solve all $\beta$ in a "scalar" fashion .
 
\item However we could use a some matrix tricks to solve for
any size of $\beta$ 

\end{itemize}
 
Implementing the equations above in SQL is fairly easy.
 
\begin{verbatim}
table
-----
X (numeric)
Y (numeric)

/**
 * m = (nSxy - SxSy) / (nSxx - SxSx)
 * b = Ay - (Ax * m)
 * N.B. S = Sum, A = Mean
 */
DECLARE @n INT
SELECT  @n = COUNT(*) FROM table
SELECT (@n * SUM(X*Y) - SUM(X) * SUM(Y)) / 
                        (@n * SUM(X*X) - SUM(X) * SUM(X)) AS M,
       AVG(Y) - AVG(X) *
       (@n * SUM(X*Y) - SUM(X) * SUM(Y)) / 
      (@n * SUM(X*X) - SUM(X) * SUM(X)) AS B
FROM table
\end{verbatim}
\section{Multiple linear regression}
Simple Linear regression alone is not enough because estimating only one parameter is very easy
and doesn't bring much benefit as opposed to estimating a vector of parameters. So scaling up the linear regression problem we bring its use to today's problems.

$y=$ 
\begin{bmatrix}
y_1 \\
y_2 \\
\vdots \\
y_n
\end{bmatrix}
$X=$  
\begin{bmatrix}
    1 & x_{11} & x_{12} & x_{13} & \dots  & x_{1p} \\
    1 & x_{21} & x_{22} & x_{23} & \dots  & x_{2p} \\
    \vdots & \vdots & \vdots & \vdots & \ddots & \vdots \\
    1 & x_{n1} & x_{n2} & x_{n3} & \dots  & x_{np}
\end{bmatrix}
$\beta=$
\begin{bmatrix}
$\beta_0$\\
$\beta_1$ \\
\vdots \\
$\beta_p$
\end{bmatrix}

$$\hat{y}=X\beta$$
$$LS(\beta) = (y - X\beta)^T(y-X\beta)$$

Minimizing the above function (taking the first derivative according to $\beta$) will yield :

$$\hat{y}=X\beta=X(X^TX)^{-1}X^Ty$$

QR Factorization simplifies the process but matrix multiplication in pure SQL
is a pain and also inneficient. The QR factorization is used by R's $lm()$ function 
(implemented with C and Fortran calls underneath).

Operations for linear algebra must be implemented :
\begin{itemize}
\item Matrix/vector multiplication
\item Matrix inverse/pseudo-inverse
\item Matrix factorization like SVD, QR, Cholesky factorization
\end{itemize}
Too much number crunching for an engine which has different purpose, even C++'s Boost library for linear algebra (BLAST) does a poor job in comparison to Matlab.



\section{Logistic Regression}
Predicts an outcome variable that is categorical from
predictor variables that are continuous and/or categorical. 
Used because having a categorical outcome variable violates
the assumption of linearity in normal regression.
It expresses the linear regression equation in logarithmic
terms (called the logit). 

In order to keep the outcome between 0 and 1, we apply the logistic function (also called sigmoid function):
$$g(z) = \frac{1}{1 + exp(-z)}$$
Bringing this to more general vector input :

$$h_\beta(x)=\frac{1}{1+e^{-\beta^Tx}}=P(y=1|x;\beta)$$

So after this function will take as input our data it will yield a "yes" or "no" answer.
And again we 

\begin{itemize}
\end{itemize}

\section{Implementation}
Simple linear regression was done efortlessly. 
The multiple linear regression and logistic regression were implemented using armadillo library.
Armadillo is a high quality linear algebra library (matrix maths) for the C++ language, aiming towards a good balance between speed and ease of use, it provides high-level syntax (API) deliberately similar to Matlab. 

$$Multiple Linear Regression$$

\begin{verbatim}
arma::colvec fastLM(const arma::vec & y, const arma::mat & X) { 
	int n = X.n_rows, k = X.n_cols;

	arma::colvec coef = arma::solve(X, y);    // fit model y ~ X
	arma::colvec res  = y - X*coef;           // residuals

	// std.errors of coefficients
	double s2 = 
	std::inner_product(res.begin(), res.end(), res.begin(), 0.0)/(n - k);

	arma::colvec std_err = 
	arma::sqrt(s2 * arma::diagvec(arma::pinv(arma::trans(X)*X)));

	return List::create(Named("coefficients") = coef,
	       Named("stderr") = std_err, 
	       Named("df.residual")  = n - k);
\end{verbatim}


\section{Conclusion}
The only kind of regression which is worth to implement directly in SQL would be the simple linear regression due to less overhead. Multiple and logistic require complex operations of a numerical nature and this is very demanding from a non turing complete language like SQL. The database engine could of course offer such statistical functionality out of the box, implemented underneath in C/C++ but for databases like Postgres, MySQL, DB2 it would be a pollution mainly because that is not their purpose.

It would be reasonable as well to ask the question : Where should such kind of calculations be performed, on the database side or server side ? 
As long as there is no complex linear algebra operations involved, than the database side is the winner, not only because it has an optimizer to aggregate data very fast but also the data must not passed between server and database.
The types of databases which could deal easily with requirements like Sqream or Kinetica which are tailored for big data queries. 



\begin{thebibliography}{9}
\bibitem{latexcompanion} 
Michel Goossens, Frank Mittelbach, and Alexander Samarin. 
\textit{The \LaTeX\ Companion}. 
Addison-Wesley, Reading, Massachusetts, 1993.
 
\bibitem{einstein} 
Albert Einstein. 
\textit{Zur Elektrodynamik bewegter K{\"o}rper}. (German) 
[\textit{On the electrodynamics of moving bodies}]. 
Annalen der Physik, 322(10):891–921, 1905.
 
\bibitem{knuthwebsite} 
Knuth: Computers and Typesetting,
\\\texttt{http://www-cs-faculty.stanford.edu/\~{}uno/abcde.html}
\end{thebibliography}



\end{document}